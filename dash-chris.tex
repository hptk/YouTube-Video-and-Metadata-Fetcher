\section{DASH}

YouTube makes use of the DASH protocol to serve its videos.

Pros and cons of HTTP crap Can be used with normal HTTP servers? DASH
makes it convenient to provide media content to users since it enables
content delivery from standard HTTP servers to HTTP clients. HTTP
provides reliable transfer of data, and enables caching of content by
standard HTTP caches. Since DASH is using HTTP as a transport protocol
it inherits many advanced features such as redirection, authentication,
traversing of NATs/firewalls, and TLS. Media resources are referred to
by using HTTP URLs, this provides a unique location for the resources,
and a simple and well-tested (?) method of accessing the resources using
HTTP GET and HTTP partial GET requests.



\subsubsection{Links}\label{links}

http caching:
https://developers.google.com/web/fundamentals/performance/optimizing-content-efficiency/http-caching?hl=en
RFC6381: https://tools.ietf.org/html/rfc6381
https://en.wikipedia.org/wiki/ISO\_8601\#Durations
https://tech.ebu.ch/docs/events/webinar043-mpeg-dash/presentations/ebu\_mpeg-dash\_webinar043.pdf
http://www.w3.org/2010/11/web-and-tv/papers/webtv2\_submission\_64.pdf
