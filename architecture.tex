\section{Architecture of the tool}
A client-server model with a REST API was chosen as the tool's architecture. The
proposed solution allows the tool to be used by multiple users simultaneously,
as well as enabling deployment on distributed systems with distinct roles,
responsibilities, and hardware resources.
The frontend is a simple thin client whose sole job is relaying commands to the
server's backend and showing the returned results. The server is tasked with obtaining
and processing the available data from YouTube, and otherwise interacting with the
API as described later.
Local deployment is a viable option, given sufficient storage and networking
capabilities.


\subsection{Client}
The user interface of the tool is written in
AngularJS\footnote{https://angularjs.org/}, an open-source framework that
facilitates easy setup and management of single-page web applications using the
model-view-controller (MVC) pattern.

The frontend is separated into four main pages, each serving a specific
function: Management of API keys, creation of search queries, creation of tasks,
and results presentation.

\subsubsection{User management}
Due to the potential usage by multiple users, a rough user management is the 
foundation to handle concurrent sessions and a login is required to access the pages
described as in the following. All the data in the database can be grouped by
a single user.

\subsubsection{API key management}
The API key management page lets the user register API keys to his account.
API keys are instantly validated and, if valid, added to the dropdown list of
available keys on the query builder page.
While only one key is required per API request (in the absence of OAuth
2.0), having access to multiple extends the tool's features to generate keys 
for a specific purpose or dataset.

\subsubsection{Query Builder}
On the query builder page the user may build individual search queries using
the provided interface. YouTube's API defines a set of options which we present
to the user in the form of input boxes. Our algorithm for assuring unique
results requires a timeframe, thus the two relevant query fields are required - 
all other fields are deemed optional. Query fields that specify operations
specifically on the user's own videos are deliberately omitted as this is
not within the scope of the tool.

Before queries are dispatched to the task workers and any real work is
performed, they are validated by issuing a complete query for syntax
verification only. The user is immediately notified if any of the given query
parameters create an invalid combination. This will prove invaluable to new
users who can safely learn to use the YouTube API, as well as preventing
the storage of invalid queries.

\subsubsection{Task Page}
Stored queries may be used to execute tasks on the Task page. The user selects
the task he or she wants performed and a query which defines the set of videos
on which to operate. The first task should always be to fetch video IDs for
the given query, as the rest of the available tasks depends on the video ID. 
Multiple tasks may be launched in parallel, and their progress is tracked in
real time.

\subsubsection{Result page}
The Result page contains selected statistics for the dataset returned by a given
query. Of particular note is the table showing the intersection between datasets.
This allows the user to quickly identify closely related queries, and can be
helpful in e.g. parameter studies. The result page can show different statistics,
and the different parts are loaded asynchronously on demand. It is relatively
easy to expand this page by adding more queries or statistics.

\subsection{Server}
The server is written in python, using flask\footnote{http://flask.pocoo.org/}
as the server framework. Flask integrates with 
Jinja\footnote{http://jinja.pocoo.org/docs/dev/templates/} for templating, 
and has support for a range of third party plugins to handle everything from
encryption to database integration. The communication between the server and
the client is established by a REST API. 


\subsubsection{Background tasks}
All requests created by the frontend are handled asynchronously by Celery and
Redis.

Celery\footnote{http://www.celeryproject.org/} is used as an asynchronous task
queue based on distributed messages. It is capable of distributing tasks over a
potentially vast network of nodes.

Redis\footnote{http://www.redis.io} is a networked in-memory key-value database.
After a task has been scheduled by the user, it is immediately pushed to Redis
and put in a pending state until a Celery worker is available to process it.
While the task is executing, Celery provides an interface to update the current
task's state. It is because of this feature an in-memory database to keep track of running
tasks preferential. Given that our fetcher modules, the actual tasks that are
being run, update their own status after every single request to
the YouTube API, a disk I/O-bound database would potentially severely limit
performance.

The frameworks are not used to their full extent in our current single-server
environment, but having these frameworks already present will be of great aid in
possible future expansion. 


