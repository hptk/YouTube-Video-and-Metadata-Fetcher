\begin{abstract}
YouTube provides a massive amount of publicly available audiovisual content.
Analyzing this data can provide invaluable information, and YouTube
is especially interesting as there is a similarly massive amount of related metadata
available. Analyzing video together with metadata such as recording time and
location, view count, like and dislike counts, description, tags and last but
not least comments enables a whole new depth of video analysis. With this in 
mind we have developed a tool that makes it easy to download a set of videos
with related metadata and comments. The tool groups the data in an SQL database.
When downloading data our tool tries to be as unbiased as possible, by
downloading as much data as it can and leaving the query parameters in the users
control. Being unbiased is important because having a representative video set
to, for instance, train a machine learning video analysis algorithm is essential
for that algorithm's ability to understand a video outside of the training set.
During development we discovered interesting aspects about the YouTube API.
The API seems to be inconsistent in the replies it gives, the API endpoint gives
slightly different results when queried with the same query multiple times. This
behavior is not unexpected - and negligible. It does not compare to the issue
that is the massive decline in returned results when stepping backwards in time.
Compared to requesting videos from yesterday, requesting videos from a day a
months back return only about 2\% of the unique video ID count. 

YouTube uses Dynamic Adaptive Streaming over HTTP (DASH) to provide its media
content. Our tool uses this to provide the user with information about available
qualities for a given video or video query set, and subsequently issues HTTP GET
requests as described in the DASH Media Presentation Description (MPD). 
YouTube's implementation of DASH does not differ much from ISO
23009-1~\cite{iso-dash-2014} where DASH is defined.
\end{abstract}
