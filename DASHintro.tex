\subsection{DASH}
DASH is an abbrevation for Dynamic Adaptive Streaming over HTTP, and is
defined in ISO/IEC 23009-1. The goal of DASH is to improve the user
experience while streaming media content, by providing easy access to
different media qualities so that the client can switch qualities based
on available bandwidth to avoid stuttering and buffering.

The core of DASH is the Media Presentation Description (MPD), sometimes
referred to as the DASH manifest. The MPD contains all the information
needed to display the media. When a user wants to access a streamed
media, for instance a video, the streaming client (or website frontend)
makes a request to get the MPD for the requested media. The client then
parses the MPD and meassures available network and buffer resources,
before the streaming is commenced at the best feasable quality. The
client continues to monitor the available resources while playing the
media back and requesting more, adapting quality dynamially if conditions
should change.

With this approach the descision making is left to the client alone. The
server only presents all available information about the media in the
MPD, it is the client that chooses how the media is to be presented to
the user. The client can choose to focus on continously playing the
media, for instance by jumping to lower qualities if the network
conditions get worse, or focus on high quality with the possible
disadvantage of stopping periodically to buffer data. The default
behaviour for most streaming clients is to prioritise continous playback,
while leaving an option for the user to manually pick quality form a list
of available qualities to override the default.

The developed tool only downloads the media files as specified by the
user, and does not care about available resources or the user experience.
The DASH manifest is still parsed, and all relevant data is stored in the
database to provide an overview of the available medias and qualities,
and their respective properties. Although we are not using DASH to
accomplish better user experience, the MPD is still very useful as a
single point to get information about a streamed media.

