\section{Future work}
Perhaps the most obvious issue to address in future work is the inconsistancy
of the YouTube API. Doing research on what causes inconsistent replies could
initially improve the effectivity of the developed tool, but perhaps more 
importantly it could be used to improve the YouTube API. 

Investigating why the API return rate drops dramatically at certain intervals
when looking back in time, and looking at what kind of videos that disappear
from the returned set, will give interesting knowledge about the YouTube API,
as well as perhaps give a final answer to whether the returned video set is
indeed biased.

There are many conceivable additions to the tool as it is now. Improving the
statistics view could improve the user experience and make it easier for the
user to make decisions about what to do next. A feature that allows the user to
export a database containing only a selected set of videos could drastically
reduce SQL query times during analysis, and we would recommend making some
feature like this if SQL querying becomes a major time and resource consumer.

There is a lot of room for performance improvements, especially when it comes to
concurrency. The tool was tested on a variety of platforms, and performed
reasonably well on most of them, but on certain configurations the API requests
were inexplicably horrendously slow. The cause of this might be a weird
combination of hardware, drivers, and TCP implementation in the OS kernel - 
we simply had no time to narrow down the possible causes. Adding support to run
multiple Celery jobs per task might have alleviated some of the performance
hangups.

