Perhaps the most obvious issue to address in future work is the unrealiability
of the YouTube API. Doing research on what causes inconsistent replies could
initially improve the effectivity of the developed tool, but perhaps more 
importantly it could be used to improve the YouTube API. 

Investigating why the API return rate drops dramatically at certain intervals
when looking back in time, and looking at what kind of videos that disappear
from the returned set, will give interesting knowledg about the YouTube API,
as well as perhaps give a final answer to wether the returned video set is
biased or not.

There are many conceivable additions to the tool as it is now. Improving the
statistics view could improve the user experience and make it easier for the
user to make decisions about what to do next. A feature that allows the user to
export a database containing only a selected set of videos could drastically
reduce SQL query times during analysis, and ww would recommend making some
feature like this if SQL querying becomes a major time and resource consumer.

