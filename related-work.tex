\section{Related Work}
A lot of work has by now been done related to DASH and Adaptive streaming in
general. DASH ~\cite{iso-dash-2014} was the first adaptive streaming
technology using HTTP that became an international standard. This is in large
due to pioneering work from Adobe with their Adobe Systems HTTP Dynamic
Streaming ~\cite{adobe-http-dynamic-streaming}, Apple with HTTP Live Streaming
(HLS) ~\cite{apple-http-streaming}, and Microsoft with Microsoft Smooth
Streaming ~\cite{microsoft-smooth-streaming}, DASH has quickly spread, and
YouTube has now implemented DASH as their preferred streaming technology
~\cite{Google I/O 2013}. 
D. Krishnappa et al. (2013) ~\cite{dashing-youtube} discussed possible resource
savings of implementing DASH in YouTube, somewhat related to our work. 

youtube-dl (sic) ~\cite{youtube-dl} is an open source tool developed by P.
Hagemeister et al. (2008-today) that provides the ability to download videos
from youtube and other similar media hosting sites. Their source code has been
the only significant inspiration for our own source code. We had the option of
simply including their source code and interact with it directly, as a way to
outsource the video file downloading, but the code required is not complex
enough to justify the loss of control. One of the most usable features
youtube-dl provides, that our tool currently does not provide, is the merging of
sound and video into a single file. This would make storage on our part a bit
less complicated, but on the other hand having split video and sound files can
be an advantage for analysis tools. 
