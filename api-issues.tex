\section{API Issues}
We have experienced that the API in some situations returns and indicates that
there are multiple pages, even if there are less then 50 videos in the result
list. This is in itself illogical, as a more natural way to return results would
be to fill each page, and only when there are less than 50 videos remaining that
match the query return a non-full list. Our experience why querying for the next
page when the list is not full, is that the page and subsequent pages contains 
the exactly same videos. Thus iterating over the pages a single static parameter
set has becomes a bad idea because it is impossible to know if there are new
videos on the next page or not.  %TODO expand on this?

We have also found that when issuing the same request to the server several
times, the API might respond slightly differently each time. This result is
easily reproducable by selecting a standard query, and seems to be more
frequent as the result list grows in size. %TODO more here!

As a part of our experiment we issued a lot of requests to the YouTube API to
gather data. A strange pattern quickly emerged, requests for timeslots longer
back than a few days contained drastically less returned unique videos, and 
stepping a month back, a request for all videos uploaded a given day will be 
about 98\% less than a request for videos uploaded yesterday. This dropoff in
returned video IDs seem to go in steps, as visualised in 
~\ref{october-video-ids}. For the past three days the returned unique video IDs
are around 200,000. The next five days back returns around 65,000 to 75,000
video IDs, and the subsequent 12-14 days return 20,000 to 30,000 videos. From
that point and backwards the number of videos per day remain stable at aorund
2,000 to 4,000, slowly declining to only a few hundred returned videos per day.

At September 26th 2015 the API returned a total 246,106 video IDs for that day.
For the same day, if queried a little more than a month later, the API returns
a mere 4,394 videos. 

This behaviour is very hard for us to explain. In 


