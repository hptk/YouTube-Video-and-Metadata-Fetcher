\section{DASH}\label{dash}

YouTube uses the DASH protocol to serice is users with videos.

Pros and cons of HTTP crap Can be used with normal HTTP servers? DASH
makes it convenient to provide media content to users since it enables
content delivery from standard HTTP servers to HTTP clients. HTTP
provides reliable transfer of data, and enables caching of content by
standard HTTP caches. Since DASH is using HTTP as a transport protocol
it inherits many advanced features such as redirection, authentication,
traversing of NATs/firewalls, and TLS. Media resources are referred to
by using HTTP URLs, this provides a unique location for the resources,
and a simple and well-tested (?) method of accessing the resources using
HTTP GET and HTTP partial GET requests.

One of the core features of DASH is the ability to request different
qualities for each video. For many YouTube videos you can choose from a
range of qualities between 122p and 1080p. Recently, YouTube also added
4k videos.

This is an example of a Representation of a 1080p mp4 video. The @id
field specifies an identifier for this Representation, it's used to do
what exactly? Each id is linked with a specific type of video. 137 is
always a 1920x1080 mp4 video, for example
{[}https://github.com/rg3/youtube-dl/blob/master/youtube\_dl/extractor/youtube.py{]}.

The @codecs field shall specify the codecs present with this
Representation. The field should also include the profile and level
information where applicable. For this video, the codec specifies a
H.264/AVC video, High Profile, Level 40 (fix).

The @width and @height fields specifies the resolution of the video in
pixels (not the ISO DASH definition, but always true for youtube
videos?).

TODO: describe SAP

@maxPlayoutRate specifies the maximum playout rate as a multiple of the
regular playout rate, in this example it is set to 1, which means that
it's not supported on any level.

The @bandwidth field is a little more complicated than the other fields.
If a Representation is continuously delivered at this bitrate (in a
constant bitrate channel of @bandwidth bps), starting at SAP 1, a client
can be assured of having enough data for continuous playout providing
playout begins after @minBufferTime * @bandwidth bits have been
received. If you consider the value to be bits per second in a channel
with constant bitrate,

Not all identifiers are specified in the ISO DASH standard. YouTube
provides some of its own, and these are prefixed with yt:. One example
is the @yt:contentLength field. This specifies the size of the
Representation in bytes. So the total download size of the
Representation will match this value.

BaseURL contains the contentLength and a HTTP URL to be used as a base
URL for the Representation.

When switching between different qualities, the base URL is used
together with content length and stuff to start downloading at the
correct loaction for the next Representation. \emph{super pr0
description here}

\begin{verbatim}
{
    "@id": "137",
        "@codecs": "avc1.640028",
        "@width": "1920",
        "@height": "1080",
        "@startWithSAP": "1",
        "@maxPlayoutRate": "1",
        "@bandwidth": "4133205",
        "@frameRate": "24",
        "BaseURL": {
            "@yt:contentLength": "148765820",
            "#text": "http://r8---sn-uxaxovg-vnad.googlevideo.com/videoplayback?id=9d3e9e6819bcd9b4&itag=137&source=youtube&ms=au&pl=22&mv=m&mn=sn-uxaxovg-vnad&mm=31&ratebypass=yes&mime=video/mp4&gir=yes&clen=148765820&lmt=1443591699166739&dur=531.864&fexp=9405989,9408209,9408710,9414764,9414930,9415870,9416126,9416179,9416984,9417132,9417707,9420934,9421175,9422460,9422592,9422596,9422674,9422867,9423429&sver=3&key=dg_yt0&upn=rTJyK8MSOVI&signature=466BA9528939E220DBE518CD8F8D00C971D3D818.0D8F0B72BC71EDC25B5CDFC9285B6A182811F218&mt=1446290848&ip=95.34.86.97&ipbits=0&expire=1446312558&sparams=ip,ipbits,expire,id,itag,source,ms,pl,mv,mn,mm,ratebypass,mime,gir,clen,lmt,dur"
        },
        "SegmentBase": {
            "@indexRange": "711-1942",
            "@indexRangeExact": "true",
            "Initialization": {
                "@range": "0-710"
            }
        }
}
\end{verbatim}

\subsubsection{Links}\label{links}

http caching:
https://developers.google.com/web/fundamentals/performance/optimizing-content-efficiency/http-caching?hl=en
RFC6381: https://tools.ietf.org/html/rfc6381
https://en.wikipedia.org/wiki/ISO\_8601\#Durations
https://tech.ebu.ch/docs/events/webinar043-mpeg-dash/presentations/ebu\_mpeg-dash\_webinar043.pdf
http://www.w3.org/2010/11/web-and-tv/papers/webtv2\_submission\_64.pdf
