% About YouTube
%	Short description
%	Why one would want an unbiased dataset
%		What does "unbiased" mean in this context
%		How have we interpreted this, and how has it affected our work?
% Our tool
%	Short description
%	How we select videos
%	User choices in request vs. biased selection
%	API over crawling
% Sections

YouTube is without a doubt the world's largest host of user-generated content,
with over one billion users generating several billion views, spending hundreds
of billions of hours every day ~\cite{officialstats}. Though there does not seem
to be an official source, it is believed that by the end of 2014, more than 300
hours worth of content was being uploaded to YouTube every minute 
~\cite{dagensmediastats}~\cite{reelseostats}. 

This makes YouTube a fantastic resource of both videos and metadata to be used
for analysis for a range of different purposes. Especially with regards to
machine learning algorithms, a large, good sample set of videos will be required
to train an algorithm. For such an algorithm to stay relevant also for videos
that was not in the inital training set, the training set will have to be as
unbiased as possible.

It is neither our desire nor task to create a tool that by default limits the
returned data set in any way. By letting the users specify as little or as much
as they want in their query, we leave as much control as possible with the user,
who, of course, knows much better than us what the resulting dataset will be
used for. This makes the tool very versatile, as you could, for instance, first
download a big set of videos related to the search term "cat", before
downloading a completely random video set and using an algorithm trained with
the first data set to find cat-related videos in this second set. 

More specifically, we provide the means to: build a large database of video ID's
\footnote{We fetch as much data as possible within the restraints imposed
on us by the API.}; fetch most metadata\footnote{We fetch all data associated
with a video, as well as its comment threads and replies. Fetching of related
videos has been deliberately left out - for now at least} for the given videos;
fetch the videos themselves, with sound; and connect all related data points
with a SQL database. Alongside the documentation for the source code there will
be a databse diagram show how all the data is related. Making an SQL database
made sense because, as we saw our dataset grow with hundreds of thousands of
videos with related metadata and comment data, storing it in CSV, XML or some
other file format made little sense. SQL is also a widely adapted database
format, and all the widely adapted programming languages has support for
extracting data from an SQL database in one way or another.

One of the main goals of the tool, as described earlier and in even more detail
later, is to be unbiased. In the context of this paper, to be unbiased means
to not whiegh videos differently based on their properties. When gathering a
set of videos, the only limitations, if any, are the ones provided by the user.
The resulting videoset will consist of all videos matching the query, without
being wheighted towards popular videos, new videos, high quality videos, 
advertised videos, recommended videos or any other imaginable parameter. This
is inherently different from the video set a normal user sees while browsing, 
and this is discussed later.

With this foundation our work has been centered around trying to gather as much
information as possible from YouTube whilst being unbiased and efficient with
resources. Resources can here mean several things, as well as the users CPU,
network and storage resources, we also have tried minimising the API quota
\footnote{A form of cradit assosiated with an API key, quota is lost when making
API requests.} costs for the user.


