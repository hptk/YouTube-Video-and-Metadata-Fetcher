% About YouTube
%	Short description
%	Why one would want an unbiased dataset
%		What does "unbiased" mean in this context
%		How have we interpreted this, and how has it affected our work?
% Our tool
%	Short description
%	How we select videos
%	User choices in request vs. biased selection
%	API over crawling
% Sections

\section{Introduction}

YouTube is without a doubt the world's largest host of user-generated content,
with over one billion users generating several billion views, spending hundreds
of billions of hours every day~\cite{officialstats}. Though there does not seem
to be an official source, it is believed that by the end of 2014, more than 300
hours worth of content was being uploaded to YouTube every
minute~\cite{dagensmediastats}~\cite{reelseostats}. 

This makes YouTube a fantastic resource of both videos and metadata intended for
uses such as e.g. statistical analysis and machine learning: Fields like these
require considerable amounts of data to be expected to yield reasonable results.
Of particular note is the relation between the videos and the corresponding 
metadata, as it may provide the means to interpret the media data in its true
context.

It is neither our desire nor task to create a tool that by default limits the
returned data set in any way. By letting the users specify as little or as much
as they want in their query, we leave as much control as possible with the user,
who has a more intimate knowledge about the dataset he wants to obtain.
This makes the tool versatile, as you could first
download a big set of videos related to the search term "cat", before
downloading a completely random video set and using an algorithm trained with
the first data set to find cat-related videos in this second set. 

More specifically, we provide the means to: Build a large database of video IDs
\footnote{We fetch as much data as possible within the restraints imposed
on us by the API.}; fetch most metadata\footnote{We fetch all data associated
with a video, as well as its comment threads and replies. Fetching of related
videos has been deliberately left out - for now at least} for the given videos;
fetch the videos themselves, with sound; and connect all related data points
with a SQL database. Alongside the documentation for the source code there will
be a database diagram to show how all the data is related.

As we tested our tool and our working data set grew up to hundreds of thousands,
if not millions of entries, we realized we could not simply store the resulting
datasets in a straight-forward manner. We needed a system that could handle the
potentially massive I/O spikes our tool would create. We went for a relational database
solution that can both handle the expected stress and facilitate connecting
related data. SQL is a widely adapted database format, supported by plugins
(both official and unofficial) in almost every programming language there is,
and it can also handle the incredible stress we will subject it to. Additionally,
it makes it easy for even a novice user to extract meaningful data from the
database.

One of the main goals of the tool, as described earlier and in even more detail
later, is to be unbiased. In the context of this paper, to be unbiased means
to not wheigh videos differently based on their properties. When gathering a
set of videos, the only limitations, if any, are the ones provided by the user.
The resulting videoset will consist of all videos matching the query, without
being wheighted towards popular videos, new videos, high quality videos, 
advertised videos, recommended videos or any other imaginable
parameter\footnote{Assuming the query responses themselves are unbiased}. This
is inherently different from the video set a normal user sees while browsing.

With this foundation our work has been centered around trying to gather as much
information as possible from YouTube whilst being both unbiased and efficient
resource-wise: We attempt to minimize CPU load, network usage, storage required,
while also minimizing the API quota\footnote{A form of credit assosiated with an API key, quota is spent making API requests.} costs for the user.

