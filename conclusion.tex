\section{Conclusion}
This article presented the YouTube implementation of DASH compared to 
ISO/IEC 23009-1. It gave an overview of the YouTube Data API v3, outlined
the major limitations of it and also provided some guidelines on how to
bypass these limitations in order to obtain a good sample of YouTube
videos for further analyzes. The main decision criteria on implementing
the searching for videos were concurrency and maximization of the amount
of responded videos while minimizing the quota costs and requests needed
to achieve this. While creating the tool, there were 2305 open issues
associated to the YouTube API on the Google Data APIs issue tracking
platform ~\cite{conclusion:gdataissue} and some of them were related to
the behaviors we have experienced.
