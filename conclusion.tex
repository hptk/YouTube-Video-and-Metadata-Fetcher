\section{Conclusion}
This paper presented the YouTube implementation of DASH. The paper presented how
YouTube distributes its media content through the DASH protocol, and how it
differs from the ISO specification of DASH. It gave an overview of the YouTube
Data API v3, outlined the major limitations of it and also provided some
guidelines on how to bypass these limitations in order to obtain a good sample
of YouTube videos for further analysis. The main decision criteria on
implementing the searching for videos were maximization of the amount of
responded videos while minimizing the quota costs and requests needed to achieve
this, while at the same time trying to maximize resource usage through
concurrency. While creating the tool, there were 2305 open issues associated to
the YouTube API on the Google Data APIs issue tracking
platform~\cite{conclusion:gdataissue} and some of them were related to the
behaviors we have experienced.
